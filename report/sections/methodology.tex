\subsection{Pure Python Implementation}
The pure Python implementation focuses on clarity and maintainability. 
The functionality of the package is implemented in a base class called Dual.
The Dual class takes two arguments: the real and the dual component, 
which is then passed to the Dual object to be used to perform the operations.

\begin{lstlisting}[language=Python]
class Dual:
    def __init__(self, real, dual=0.0):
        self.real = real
        self.dual = dual
\end{lstlisting}

Basic operations such as addition, subtraction, multiplication, division, and indices are implemented as methods of the Dual class. 
These operations were overloaded over standard symbols such as $+$, $-$, $*$, $/$, and $**$.
The reverse operators (for when the Dual class is on the right-hand side of the operator), were also defined for completeness. 
In the case of a Dual number and a non-Dual number, the operator will turn the non-Dual number into a Dual number, 
with a Dual component of 0. The operations will always return a Dual object. 

An example of overloading the $+$ operator:

\begin{lstlisting}[language=Python]
    def __add__(self, other) -> "Dual":
    # Check if the other object is a Dual number
    if not isinstance(other, Dual):
        other = Dual(other)
    return Dual(self.real + other.real, self.dual + other.dual)
\end{lstlisting}

For two dual numbers: $a + b\epsilon$, $c + d\epsilon$, the following mathematical operations were defined:

\subsubsection{Addition}
Addition is simply just the addition of the real and Dual components.

\begin{equation}
    (a + b\epsilon) + (c + d\epsilon) = (a + c) + (b + d)\epsilon
    \label{eq:addition}
    \end{equation}

\subsubsection{Subtraction}
Subtraction is simply the subtraction of the real and Dual components.

\begin{equation}
    (a + b\epsilon) - (c + d\epsilon) = (a - c) + (b - d)\epsilon
    \label{eq:subtraction}
    \end{equation}

\subsubsection{Multiplication}
Multiplication of two dual numbers involves multiplying the real parts and applying the distributive property to the dual parts:

\begin{equation}
    (a + b\epsilon) \times (c + d\epsilon) = (a \times c) + (a \times d + b \times c)\epsilon
    \label{eq:multiplication}
    \end{equation}

\subsubsection{Division}
Division of two dual numbers involves dividing the real parts and applying the quotient rule to the dual parts:

\begin{equation}
    \frac{a + b\epsilon}{c + d\epsilon} = \frac{a}{c} + \frac{b \times c - a \times d}{c^2}\epsilon
    \label{eq:division}
\end{equation}

\subsubsection{Exponentiation}
Raising a dual number to the power of another dual number involves using the chain rule and logarithmic identities:

\begin{equation}
    (a + b\epsilon)^{(c + d\epsilon)} = a^c \left(1 + \left(\frac{b \cdot c}{a} + d \cdot \ln(a)\right)\epsilon\right)
    \label{eq:exponentiation}
\end{equation}

A number of useful functions were also added to the base Dual class:

\subsubsection{Sine Function}
The sine of a dual number is calculated by applying the sine function to the real part and the cosine function to the dual part:

\begin{equation}
    \sin(a + b\epsilon) = \sin(a) + b \cdot \cos(a)\epsilon
    \label{eq:sine}
\end{equation}

\subsubsection{Cosine Function}
The cosine of a dual number is calculated by applying the cosine function to the real part and the negative sine function to the dual part:

\begin{equation}
    \cos(a + b\epsilon) = \cos(a) - b \cdot \sin(a)\epsilon
    \label{eq:cosine}
\end{equation}

\subsubsection{Tangent Function}
The tangent of a dual number is calculated by applying the tangent function to the real part and the derivative of the tangent function to the dual part:

\begin{equation}
    \tan(a + b\epsilon) = \tan(a) + b \cdot \sec^2(a)\epsilon
    \label{eq:tangent}
\end{equation}

\subsubsection{Logarithm Function}
The logarithm of a dual number is calculated by applying the logarithm function to the real part and dividing the dual part by the real part:

\begin{equation}
    \log(a + b\epsilon) = \log(a) + \frac{b}{a}\epsilon
    \label{eq:logarithm}
\end{equation}

\subsubsection{Exponential Function}
The exponential of a dual number is calculated by applying the exponential function to the real part and multiplying the dual part by the exponential of the real part:

\begin{equation}
    e^{(a + b\epsilon)} = e^a \left(1 + b\epsilon\right)
    \label{eq:exponential}
\end{equation}

\subsubsection{Differentiation}
The crucial use of Dual numbers is in being able to efficiently calculate the exact derivative of a function at a certain point.
Given a function: $f(x)$, the derivative at point $x_0$ is calculated with Dual numbers in the following way:

Expand $f(x_0 + \epsilon)$ using a Taylor Expansion

\begin{equation}
    f(x_0 + \epsilon) = f(x_0) + f'(x_0) \epsilon + \frac{f''(x_0)}{2} \epsilon^2 \dots
    \label{eq:taylor_exp}
\end{equation}

Since $\epsilon^2 = 0$, all of the higher order terms vanish leaving:

\begin{equation}
    f(x_0 + \epsilon) = f(x_0) + f'(x_0) \epsilon
    \label{eq:taylor_exp_no_higher}
\end{equation}

Therefore, the derivative of $f(x)$ at point $x_0$ is the coefficient of the dual part of the result, 
$f'(x_0) \epsilon$

\subsection{Cythonised Implementation}
Cythonisation is the process of compiling Python code in C, 
which can significantly speed-up execution, especially for numerical computations and loops. \\

A separate module called \texttt{dual\_autodiff\_x} was created for the Cython code, 
where a new base class was created (\texttt{dual.pyx}), with the exact same functionality as the pure Python package.
In this base class, static type declaration was performed for the class and its functions.

\begin{lstlisting}[language=Python]
cdef class Dual:
    cdef public double real
    cdef public double dual
    
    def __init__(self, double real, double dual=0.0):
        self.real = real
        self.dual = dual
\end{lstlisting}

The Cythonisation of the code was performed using the \texttt{Cython} 
library for the build process (compiling the \texttt{dual.c} file)

\subsection{Build System and Packaging}
\subsubsection{Packaging}
To perform the setup for the package, a \texttt{pyproject.toml} was included the code. 
It contains all of the project information, dependencies, 
and tools for automatic versioning, and packaging. 
It allows the package to be installed on any environment by running 
\texttt{pip install -e} from the root folder of the module.  
The Cythonised package had its own \texttt{pyproject.toml} and \texttt{setup} file, 
so that it could be installed as a standalone package if required by the user. 

\subsubsection{Git, CI/CD and Test Suite}
For integration with Git, and to perform Continuous Integration and Continuous Development (CI/CD),
the package comes fully tested (testing both valid and invalid inputs). This test suite prevents incorrect changes to the code. 
This is integrated by creating a automatic pre-commit git hook. 
Automatic versioning is also implemented, where the version is updated for every push to the main branch.
Due to the package structure, the \texttt{dual\_autodiff\_x} does not have automatic versioning,
as it would require a fully-built Git repo, which would lead to a Git repo within a Git repo.

\subsubsection{Compiling Cython Wheels}
\texttt{cibuildwheel} was used to create the wheels for specific Linux and Python (3.10, 3.11) installations using different Docker images. 
Wheels offer many benefits, including pre-compiling binaries (removing the need for compiliers on machines), 
faster installation, and dependency management. It is especially necessary for Cythonsied code, to have separate
wheels as the compilers found on Linux, OSX, and Windows laptops can be different. 
Without the necessary wheels, the C code will not compile. 
The \texttt{.pyx} (source code) was excluded from the wheels, following good practice. 

\subsubsection{Documentation}
Documentation is vital for users to understand how the package works, and should always be up to date.
In this sense, the ability to automatically update documentation whenever there is a change to code is useful. 
This is implemented using the \texttt{Sphinx} library, which detects the fully documented (type-hinting and docstrings)
within the code, and outputs them to a \texttt{.html} file within the \texttt{docs/build} folder.  
Jupyter notebooks are also integrated into the documentation using the \texttt{pandoc} extension. 
By running \texttt{make html} from the root directory, the all necessary dependencies are installed, 
and the documentation is built.
The dependencies for documentation are not included in the wheels or pyproject.toml file, 
as they are not necessary for the package to run.
An improvement would be to create another Git commit hook to update the documentation when changes are mode to the code. 


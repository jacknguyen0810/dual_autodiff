The concept of automatic differentiation (AD) has become increasingly important in modern scientific computing, 
particularly in machine learning and optimization. Dual numbers, an extension of real numbers that include an 
infinitesimal component, provide an elegant approach to implementing forward-mode automatic differentiation.

The goal of this project is to implement a dual number system in pure Python and optimize performance through Cythonization. 
We will compare the performance characteristics of the two implementations and analyze the trade-offs between 
implementation approaches. These techniques are packaged in a Python package following coding good practices.

\subsection{Background}
Dual numbers, introduced by William Clifford \cite{clifford1873}, extend real numbers by adding a nilpotent element 
$\epsilon$ with the property $\epsilon^2 = 0$. A dual number takes the form:

 $a + b\epsilon$, 

where $a$ represents the real part and $b$ represents the dual part.